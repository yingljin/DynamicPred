% Options for packages loaded elsewhere
\PassOptionsToPackage{unicode}{hyperref}
\PassOptionsToPackage{hyphens}{url}
\PassOptionsToPackage{dvipsnames,svgnames,x11names}{xcolor}
%
\documentclass[
  11pt,
]{article}
\usepackage{amsmath,amssymb}
\usepackage[]{mathpazo}
\usepackage{iftex}
\ifPDFTeX
  \usepackage[T1]{fontenc}
  \usepackage[utf8]{inputenc}
  \usepackage{textcomp} % provide euro and other symbols
\else % if luatex or xetex
  \usepackage{unicode-math}
  \defaultfontfeatures{Scale=MatchLowercase}
  \defaultfontfeatures[\rmfamily]{Ligatures=TeX,Scale=1}
\fi
% Use upquote if available, for straight quotes in verbatim environments
\IfFileExists{upquote.sty}{\usepackage{upquote}}{}
\IfFileExists{microtype.sty}{% use microtype if available
  \usepackage[]{microtype}
  \UseMicrotypeSet[protrusion]{basicmath} % disable protrusion for tt fonts
}{}
\makeatletter
\@ifundefined{KOMAClassName}{% if non-KOMA class
  \IfFileExists{parskip.sty}{%
    \usepackage{parskip}
  }{% else
    \setlength{\parindent}{0pt}
    \setlength{\parskip}{6pt plus 2pt minus 1pt}}
}{% if KOMA class
  \KOMAoptions{parskip=half}}
\makeatother
\usepackage{xcolor}
\usepackage[margin = 1in]{geometry}
\usepackage{graphicx}
\makeatletter
\def\maxwidth{\ifdim\Gin@nat@width>\linewidth\linewidth\else\Gin@nat@width\fi}
\def\maxheight{\ifdim\Gin@nat@height>\textheight\textheight\else\Gin@nat@height\fi}
\makeatother
% Scale images if necessary, so that they will not overflow the page
% margins by default, and it is still possible to overwrite the defaults
% using explicit options in \includegraphics[width, height, ...]{}
\setkeys{Gin}{width=\maxwidth,height=\maxheight,keepaspectratio}
% Set default figure placement to htbp
\makeatletter
\def\fps@figure{htbp}
\makeatother
\setlength{\emergencystretch}{3em} % prevent overfull lines
\providecommand{\tightlist}{%
  \setlength{\itemsep}{0pt}\setlength{\parskip}{0pt}}
\setcounter{secnumdepth}{-\maxdimen} % remove section numbering
\newlength{\cslhangindent}
\setlength{\cslhangindent}{1.5em}
\newlength{\csllabelwidth}
\setlength{\csllabelwidth}{3em}
\newlength{\cslentryspacingunit} % times entry-spacing
\setlength{\cslentryspacingunit}{\parskip}
\newenvironment{CSLReferences}[2] % #1 hanging-ident, #2 entry spacing
 {% don't indent paragraphs
  \setlength{\parindent}{0pt}
  % turn on hanging indent if param 1 is 1
  \ifodd #1
  \let\oldpar\par
  \def\par{\hangindent=\cslhangindent\oldpar}
  \fi
  % set entry spacing
  \setlength{\parskip}{#2\cslentryspacingunit}
 }%
 {}
\usepackage{calc}
\newcommand{\CSLBlock}[1]{#1\hfill\break}
\newcommand{\CSLLeftMargin}[1]{\parbox[t]{\csllabelwidth}{#1}}
\newcommand{\CSLRightInline}[1]{\parbox[t]{\linewidth - \csllabelwidth}{#1}\break}
\newcommand{\CSLIndent}[1]{\hspace{\cslhangindent}#1}
\ifLuaTeX
  \usepackage{selnolig}  % disable illegal ligatures
\fi
\IfFileExists{bookmark.sty}{\usepackage{bookmark}}{\usepackage{hyperref}}
\IfFileExists{xurl.sty}{\usepackage{xurl}}{} % add URL line breaks if available
\urlstyle{same} % disable monospaced font for URLs
\hypersetup{
  pdftitle={Dynamic Prediction of Non-Guassian Outcome with fast Generalized Functional Principal Analysis},
  pdfauthor={Ying Jin; Andrew Leroux},
  colorlinks=true,
  linkcolor={blue},
  filecolor={Maroon},
  citecolor={Blue},
  urlcolor={Blue},
  pdfcreator={LaTeX via pandoc}}

\title{Dynamic Prediction of Non-Guassian Outcome with fast Generalized
Functional Principal Analysis}
\author{Ying Jin \and Andrew Leroux}
\date{March 07, 2023}

\begin{document}
\maketitle

\hypertarget{abstract}{%
\section{Abstract:}\label{abstract}}

Biomedical investigators are often interested in predicting future
observations of subjects based on their historical data, referred to as
dynamic prediction. Traditional methods are often limited in flexibility
and computationally intensive, especially with non-Gaussian data. To
address these issues, we propose a novel method for dynamic prediction
based on Generalized Functional Principal Component Analysis (FPCA).
Assume the observed outcome follows an exponential family distribution
parameterized by a latent Gaussian function, the proposed method
consists of the following steps: 1) Bin the data across functional
domain into small, equal-length intervals; 2) Fit local generalized
mixed models at every bin to estimate individual latent functions; 3)
Fit FPCA model to smooth latent functions and 4) Obtain estimates of
subject-specific PC scores using partial observations and recover the
unobserved part on the binned grid. Our simulation study showed the
proposed method achieved significantly better out-of-sample predictive
performance compared to existing methods with much shorter computation
time, thus has the potential to be widely applicable to large datasets.

\hypertarget{introduction}{%
\section{Introduction}\label{introduction}}

\begin{itemize}
\tightlist
\item
  Overview of dynamic prediction methods
\end{itemize}

Prediction of repeated measures has been a problem of interest in the
biomedical field. Typically, such predictions are made based on the
correlation between repeated measures from the same subject, and/or
covariates that can be either fixed or time-varying. Traditionally,
repeated measures have been modeled using marginal models (generalized
estimating equations) or conditional models (mixed effect models)
(\protect\hyperlink{ref-Laird1982}{Laird and Ware 1982};
\protect\hyperlink{ref-liang1986}{LIANG and ZEGER 1986};
\protect\hyperlink{ref-lindstrom1990}{Lindstrom and Bates 1990};
\protect\hyperlink{ref-davidian2003}{{``Nonlinear models for repeated
measurement data''} 2003};
\protect\hyperlink{ref-GLMMadaptive}{Rizopoulos 2022}). These methods,
while allowing for correlation between repeated measures, are limited in
terms of flexibility of correlation structure and the ability to handle
out-of-sample prediction. Therefore, one may turn to functional mixed
effect models when measures are dense across the domain. Such methods
accommodates more flexible correlation structure by modeling
subject-specific random effects as a function, but often cause
nontrivial computational burden. A feasible approach to address this
issue is non-parametric smoothing
(\protect\hyperlink{ref-Scheipl2014}{Scheipl et al. 2014}), such as
spline basis functions or eigenfunctions from functional principal
component analysis (fPCA). The introduction of basis functions also
makes out-of-sample prediction more straightforward. Instead of
estimating subject-specific random effects of new observations, we can
simply estimate coefficients/loadings on the basis function used for
smoothing. In this project, we will focus on the prediction of
non-Gaussian outcomes (e.g.~binary or count) from a random-intercept
only model, with no covariates considered. In other words, we aim to
propose a new fast, scalable method for dynamic prediction of discrete
function tracks based only on past observations using functional mixed
effect model with fPCA smoothing.

\begin{itemize}
\tightlist
\item
  Dynamic prediction with functional methods
\end{itemize}

Research on dynamic prediction of functional outcomes has been focusing
on continuous/Gaussian outcomes, modelling subject-specific random
effects with FPCA (\protect\hyperlink{ref-chiou2012}{Chiou 2012};
\protect\hyperlink{ref-goldberg2014}{Goldberg et al. 2014};
\protect\hyperlink{ref-shang2017}{Shang 2017}). Kraus
(\protect\hyperlink{ref-kraus2015}{2015}) has used this approach to
predict missing observations in partially observed function tracks, and
Delaigle and Hall (\protect\hyperlink{ref-delaigo2016}{2016}) achieved
similar goals using Markov Chains. While methods mentioned above used
only partial observations for prediction with an intercept-only model,
Leroux et al. (\protect\hyperlink{ref-leroux2016}{2018}) proposed
Functional Concurrent Regression (FCR) framework which can incorporate
the effect of subject-specific predictors. However, little extension was
made on prediction of non-Gaussian functions, such as binary and count
outcomes.

\begin{itemize}
\tightlist
\item
  fPCA and GFPCA (\protect\hyperlink{ref-leroux2022}{Leroux et al.
  n.d.})
\end{itemize}

Unlike FPCA on Gaussian data, fewer papers have focused on its extension
to non-Gaussian data, such as series of binary or count outcomes.
Existing methods also tend to be computationally intensive. For example,
Chen et al. (\protect\hyperlink{ref-chen2013}{2013}) proposed approaches
to fit marginal functional models that is compatible to multi-level,
generalized outcomes. Goldsmith et al.
(\protect\hyperlink{ref-goldsmith2015}{2015}) established a model
framework that takes into account the fixed effect of time-invariant
covariates, with parameters estimated with Bayesian method in
\emph{Stan}. Gertheiss et al.
(\protect\hyperlink{ref-gertheiss2016}{2016}) identified bias introduced
by directly applying FPCA methods to generalized functions, and proposed
to address this problem using a two-stage, joint estimation strategy.
Linde (\protect\hyperlink{ref-linde2019}{2009}) used an adapted Bayesian
variational algorithm for FPCA of binary and count data. In terms of
implementation, Wrobel et al. (\protect\hyperlink{ref-wrobel2019}{2019})
proposed a fast, efficient way to fit GFPCA on binary data using EM
algorithm, accompanied by the an open source R package \emph{registr}.

\hypertarget{method}{%
\section{Method}\label{method}}

\begin{itemize}
\tightlist
\item
  Need better notation system
\end{itemize}

\hypertarget{result}{%
\section{Result}\label{result}}

\begin{itemize}
\tightlist
\item
  Repeat simulation: is it necessary in this case?
\item
  Different set-up:
\end{itemize}

\begin{enumerate}
\def\labelenumi{\alph{enumi}.}
\tightlist
\item
  Different eigenfunctions: with or without periodicity
\item
  Outcome: binary or count
\item
  Sample size
\item
  Grid density
\item
  Bin width, overlap or not
\end{enumerate}

\begin{itemize}
\tightlist
\item
  Real data application
\end{itemize}

\hypertarget{discussion}{%
\section{Discussion}\label{discussion}}

\begin{itemize}
\tightlist
\item
  Grid
\item
  Score bias: cannot demonstrate without repeat simulation
\end{itemize}

\hypertarget{references}{%
\section{References}\label{references}}

\hypertarget{refs}{}
\begin{CSLReferences}{1}{0}
\leavevmode\vadjust pre{\hypertarget{ref-chen2013}{}}%
Chen, H., Wang, Y., Paik, M. cho, and Choi, H. A. (2013), {``A marginal
approach to reduced-rank penalized spline smoothing with application to
multilevel functional data,''} \emph{J Am Stat Assoc.}, 108, 1216--1229.
\url{https://doi.org/10.1080/01621459.2013.826134}.

\leavevmode\vadjust pre{\hypertarget{ref-chiou2012}{}}%
Chiou, J.-M. (2012), {``Dynamical functional prediction and
classification, with application to traffic flow prediction,''}
\emph{The Annals of Applied Statistics}, Institute of Mathematical
Statistics, 6, 1588--1614. \url{https://doi.org/10.1214/12-AOAS595}.

\leavevmode\vadjust pre{\hypertarget{ref-delaigo2016}{}}%
Delaigle, A., and Hall, P. (2016), {``Approximating fragmented
functional data by segments of markov chains,''} \emph{Biometrika}, 103,
779--799. \url{https://doi.org/10.1093/biomet/asw040}.

\leavevmode\vadjust pre{\hypertarget{ref-gertheiss2016}{}}%
Gertheiss, J., Goldsmith, J., and Staicu, A. (2016), {``A note on
modeling sparse exponential-family functional response curves,''}
\emph{Comput Stat Data Anal}, 105, 46--52.
\url{https://doi.org/10.1016/j.csda.2016.07.010}.

\leavevmode\vadjust pre{\hypertarget{ref-goldberg2014}{}}%
Goldberg, Y., Ritov, Y., and Mandelbaum, A. (2014), {``Predicting the
continuation of a function with applications to call center data,''}
\emph{Journal of Statistical Planning and Inference}, 147, 53--65.
https://doi.org/\url{https://doi.org/10.1016/j.jspi.2013.11.006}.

\leavevmode\vadjust pre{\hypertarget{ref-goldsmith2015}{}}%
Goldsmith, J., Zipunnikov, V., and Schrack, J. (2015), {``Generalized
multilevel function-on-scalar regression and principal component
analysis,''} \emph{Biometrics}, 71, 344--53.
\url{https://doi.org/10.1111/biom.12278}.

\leavevmode\vadjust pre{\hypertarget{ref-hall2018}{}}%
Hall, P., Müller, H.-G., and Yao, F. (2008), {``Modelling sparse
generalized longitudinal observations with latent gaussian processes,''}
\emph{Journal of the Royal Statistical Society: Series B (Statistical
Methodology)}, 70, 703--723.
https://doi.org/\url{https://doi.org/10.1111/j.1467-9868.2008.00656.x}.

\leavevmode\vadjust pre{\hypertarget{ref-kraus2015}{}}%
Kraus, D. (2015),
{``\href{http://www.jstor.org/stable/24775309}{Components and completion
of partially observed functional data},''} \emph{Journal of the Royal
Statistical Society. Series B (Statistical Methodology)}, {[}Royal
Statistical Society, Wiley{]}, 77, 777--801.

\leavevmode\vadjust pre{\hypertarget{ref-Laird1982}{}}%
Laird, N. M., and Ware, J. H. (1982),
{``\href{http://www.jstor.org/stable/2529876}{Random-effects models for
longitudinal data},''} \emph{Biometrics}, {[}Wiley, International
Biometric Society{]}, 38, 963--974.

\leavevmode\vadjust pre{\hypertarget{ref-leroux2022}{}}%
Leroux, A., Crainiceanu, C. M., and Wrobel, J. (n.d.). {``Fast
generalized functional principal component analysis.''}

\leavevmode\vadjust pre{\hypertarget{ref-leroux2016}{}}%
Leroux, A., Xiao, L., Crainiceanu, C., and Checkley, W. (2018),
{``Dynamic prediction in functional concurrent regression with an
application to child growth,''} \emph{Statistics in medicine}, 37,
1376--1388.

\leavevmode\vadjust pre{\hypertarget{ref-liang1986}{}}%
LIANG, K.-Y., and ZEGER, S. L. (1986), {``Longitudinal data analysis
using generalized linear models,''} \emph{Biometrika}, 73, 13--22.
\url{https://doi.org/10.1093/biomet/73.1.13}.

\leavevmode\vadjust pre{\hypertarget{ref-linde2019}{}}%
Linde, van der (2009), {``A bayesian latent variable approach to
functional principal components analysi with binary and count data,''}
\emph{A StA Adv Stat Anal}, 307--333.
\url{https://doi.org/10.1007/s10182-009-0113-6}.

\leavevmode\vadjust pre{\hypertarget{ref-lindstrom1990}{}}%
Lindstrom, M. J., and Bates, D. M. (1990),
{``\href{http://www.jstor.org/stable/2532087}{Nonlinear mixed effects
models for repeated measures data},''} \emph{Biometrics}, {[}Wiley,
International Biometric Society{]}, 46, 673--687.

\leavevmode\vadjust pre{\hypertarget{ref-davidian2003}{}}%
{``\href{http://www.jstor.org/stable/1400665}{Nonlinear models for
repeated measurement data: An overview and update}''} (2003),
{[}International Biometric Society, Springer{]}, 8, 387--419.

\leavevmode\vadjust pre{\hypertarget{ref-GLMMadaptive}{}}%
Rizopoulos, D. (2022),
\emph{\href{https://CRAN.R-project.org/package=GLMMadaptive}{GLMMadaptive:
Generalized linear mixed models using adaptive gaussian quadrature}}.

\leavevmode\vadjust pre{\hypertarget{ref-Scheipl2014}{}}%
Scheipl, F., Staicu, A.-M., and Greven, S. (2014), {``Functional
additive mixed models,''} \emph{J Comput Graph Stat}, 24, 447--501.
\url{https://doi.org/10.1080/10618600.2014.901914}.

\leavevmode\vadjust pre{\hypertarget{ref-shang2017}{}}%
Shang, H. L. (2017), {``Functional time series forecasting with dynamic
updating: An application to intraday particulate matter
concentration,''} \emph{Econometrics and Statistics}, 1, 184--200.
https://doi.org/\url{https://doi.org/10.1016/j.ecosta.2016.08.004}.

\leavevmode\vadjust pre{\hypertarget{ref-suresh2017}{}}%
Suresh, K., Taylor, J. M. G., Spratt, D. E., Daignault, S., and
Tsodikov, A. (2017), {``Comparison of joint modeling and landmarking for
dynamic prediction under an illness-death model,''} \emph{Biom J}, 59,
1277--1300. \url{https://doi.org/10.1002/bimj.201600235}.

\leavevmode\vadjust pre{\hypertarget{ref-wrobel2019}{}}%
Wrobel, J., Zipunnikov, V., Schrack, J., and Goldsmith, J. (2019),
{``Registration for exponential family functional data,''}
\emph{Biometrics}, 75, 48--57.
https://doi.org/\url{https://doi.org/10.1111/biom.12963}.

\end{CSLReferences}

\end{document}
