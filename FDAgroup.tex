% Options for packages loaded elsewhere
\PassOptionsToPackage{unicode}{hyperref}
\PassOptionsToPackage{hyphens}{url}
%
\documentclass[
  ignorenonframetext,
]{beamer}
\usepackage{pgfpages}
\setbeamertemplate{caption}[numbered]
\setbeamertemplate{caption label separator}{: }
\setbeamercolor{caption name}{fg=normal text.fg}
\beamertemplatenavigationsymbolsempty
% Prevent slide breaks in the middle of a paragraph
\widowpenalties 1 10000
\raggedbottom
\setbeamertemplate{part page}{
  \centering
  \begin{beamercolorbox}[sep=16pt,center]{part title}
    \usebeamerfont{part title}\insertpart\par
  \end{beamercolorbox}
}
\setbeamertemplate{section page}{
  \centering
  \begin{beamercolorbox}[sep=12pt,center]{part title}
    \usebeamerfont{section title}\insertsection\par
  \end{beamercolorbox}
}
\setbeamertemplate{subsection page}{
  \centering
  \begin{beamercolorbox}[sep=8pt,center]{part title}
    \usebeamerfont{subsection title}\insertsubsection\par
  \end{beamercolorbox}
}
\AtBeginPart{
  \frame{\partpage}
}
\AtBeginSection{
  \ifbibliography
  \else
    \frame{\sectionpage}
  \fi
}
\AtBeginSubsection{
  \frame{\subsectionpage}
}
\usepackage{amsmath,amssymb}
\usepackage{lmodern}
\usepackage{iftex}
\ifPDFTeX
  \usepackage[T1]{fontenc}
  \usepackage[utf8]{inputenc}
  \usepackage{textcomp} % provide euro and other symbols
\else % if luatex or xetex
  \usepackage{unicode-math}
  \defaultfontfeatures{Scale=MatchLowercase}
  \defaultfontfeatures[\rmfamily]{Ligatures=TeX,Scale=1}
\fi
% Use upquote if available, for straight quotes in verbatim environments
\IfFileExists{upquote.sty}{\usepackage{upquote}}{}
\IfFileExists{microtype.sty}{% use microtype if available
  \usepackage[]{microtype}
  \UseMicrotypeSet[protrusion]{basicmath} % disable protrusion for tt fonts
}{}
\makeatletter
\@ifundefined{KOMAClassName}{% if non-KOMA class
  \IfFileExists{parskip.sty}{%
    \usepackage{parskip}
  }{% else
    \setlength{\parindent}{0pt}
    \setlength{\parskip}{6pt plus 2pt minus 1pt}}
}{% if KOMA class
  \KOMAoptions{parskip=half}}
\makeatother
\usepackage{xcolor}
\newif\ifbibliography
\setlength{\emergencystretch}{3em} % prevent overfull lines
\providecommand{\tightlist}{%
  \setlength{\itemsep}{0pt}\setlength{\parskip}{0pt}}
\setcounter{secnumdepth}{-\maxdimen} % remove section numbering
\ifLuaTeX
  \usepackage{selnolig}  % disable illegal ligatures
\fi
\IfFileExists{bookmark.sty}{\usepackage{bookmark}}{\usepackage{hyperref}}
\IfFileExists{xurl.sty}{\usepackage{xurl}}{} % add URL line breaks if available
\urlstyle{same} % disable monospaced font for URLs
\hypersetup{
  pdftitle={Dynamic Prediction with fPCA},
  hidelinks,
  pdfcreator={LaTeX via pandoc}}

\title{Dynamic Prediction with fPCA}
\author{}
\date{\vspace{-2.5em}2022-11-3}

\begin{document}
\frame{\titlepage}

\begin{frame}{Dynamic prediction}
\protect\hypertarget{dynamic-prediction}{}
\begin{itemize}
\tightlist
\item
  With observations up to \(t_m\), predict outcomes (or probabilities of
  outcome) after that time point
\item
  Prediction updates with new observations
\item
  Mixed model prediction is difficult

  \begin{itemize}
  \tightlist
  \item
    Not flexible enough for densely measured data
  \item
    Out-of-sample random effects cannot be estimated
  \end{itemize}
\item
  Functional mixed effects model
\end{itemize}
\end{frame}

\begin{frame}{Functional Concurrent Regression (FCR)}
\protect\hypertarget{functional-concurrent-regression-fcr}{}
\begin{itemize}
\item
  Goal: to predict future track based on partially observed track
\item
  For a subject i, we observe a function over t
  \[Y_i(t)=f_0(t)+b_i(t)+\epsilon_i(t)\]
\item
  Subject-specific random effect
\end{itemize}

\[b_i(t) = \Sigma_{k=1}^c u_{ik}B_k(t)\] where
\(\boldsymbol{u_i}\sim N(0, \Gamma)\)

\begin{itemize}
\tightlist
\item
  We usually observe \(Y_i\) on a series of discrete \(t_{ij}\)
\end{itemize}

\[Y_{ij} = f_0(t_{ij})+b_i(t_{ij})+\epsilon_{ij}, \hspace{0.5cm} j = 1...J_i\]
where \(\epsilon_{ij} \sim N(0, \sigma_{\epsilon^2})\).
\end{frame}

\begin{frame}{Connection to fPCA}
\protect\hypertarget{connection-to-fpca}{}
\begin{itemize}
\item
  When there is no covariate in the model, this is essentially a fPCA
  problem.

  \begin{itemize}
  \tightlist
  \item
    \(\boldsymbol{B}\) is a matrix of eigenfunctions
  \item
    \(\boldsymbol{u}\) is a matrix of PC scores/loadings
  \end{itemize}
\item
  Use fPCA to estimate \(f_0\), \(\Gamma\) and \(\sigma_{\epsilon}\)
\item
  For a new subject with observations up to \(t_m\), estimate its score:
\end{itemize}

\[\hat{\boldsymbol{u}} = E(\boldsymbol{u}|\boldsymbol{y}) =  \boldsymbol{\hat{\Gamma}B^T}(\boldsymbol{B\hat{\Gamma}B^T}+\hat{\sigma_{\epsilon}}^2\boldsymbol{I_m})^{-1}\boldsymbol{y}\]
With the estimated score, we can predict its outcome in following time
points

\[\hat{\boldsymbol{Y}} = \hat{\boldsymbol{f}}_0+ \boldsymbol{B}^T\hat{\boldsymbol{u}} \]
\end{frame}

\begin{frame}{Extension to non-Gaussian data}
\protect\hypertarget{extension-to-non-gaussian-data}{}
\begin{itemize}
\tightlist
\item
  fPCA on non-Gaussian data is very difficult and computationally
  intensive
\item
  fPCA on a latent Gaussian variable instead

  \begin{itemize}
  \tightlist
  \item
    transformation using link function:
    \(g(E(Y))=\boldsymbol{X}^T\boldsymbol{\beta}\)
  \end{itemize}
\item
  Get a smooth latent Gaussian function by pooling a series of GLMM

  \begin{itemize}
  \item
    Assume we have regularly observed functions \(Y_{ij}\),
    \(i = 1... N\) and \(j = 1...J\)
  \item
    Bin the functional domain into small intervals \(t\in {1...T}\)
  \item
    Fit a GLMM in each bin to get a latent variables

    \[g(E(Y_{it})) = \beta_{0t}+b_{it}\] Let \(Z_{it} = g(E(Y_{it}))\),
    we can do Gaussian fPCA on this latent variable
  \end{itemize}
\end{itemize}
\end{frame}

\begin{frame}{Extension to non-Gaussian data}
\protect\hypertarget{extension-to-non-gaussian-data-1}{}
\[Z_i(t) = a_{i1}sin^2(t)+a_{i2}cos^2(t)+a_{i3}t^2+a_{i4}t\]

\[Y_i(t) \sim Binomial(\frac{exp(Z_i(t))}{1+exp(Z_i(t))})\]
\end{frame}

\begin{frame}{Next steps}
\protect\hypertarget{next-steps}{}
\begin{itemize}
\tightlist
\item
  Simulate a non-Gaussian function and implement methods above

  \begin{itemize}
  \tightlist
  \item
    Improve estiation of latent functions very well
  \item
    Numeric problems
  \end{itemize}
\item
  Transform latent function back to non-Gaussian function
\item
  Inclusion of covariates
\end{itemize}
\end{frame}

\end{document}
